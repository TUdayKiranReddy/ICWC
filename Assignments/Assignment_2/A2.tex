\documentclass{article}
\usepackage[utf8]{inputenc}
\usepackage{tabularx} % extra features for tabular environment
\usepackage{amsmath}  % improve math presentation
\usepackage{graphicx} % takes care of graphic including machinery
\usepackage{xspace}
\usepackage{tikz}
\usepackage{enumitem}
\usetikzlibrary{babel}
\usepackage[american]{circuitikz}
\usetikzlibrary{calc}
\usepackage{float}
\usepackage{siunitx}
\usepackage{pgfplots}
\usepackage[skins,theorems]{tcolorbox}
\tcbset{highlight math style={enhanced,
  colframe=red,colback=white,arc=0pt,boxrule=1pt}}
\pgfplotsset{width=10cm,compat=1.9}
\usepackage[margin=1in,letterpaper]{geometry} % decreases margins
\usepackage{cite} % takes care of citations
\usepackage[final]{hyperref} % adds hyper links inside the generated PDF file
\hypersetup{
colorlinks=true,       % false: boxed links; true: colored links
linkcolor=blue,        % color of internal links
citecolor=blue,        % color of links to bibliography
filecolor=magenta,     % color of file links
urlcolor=blue        
}

\begin{document}

\title{{\textbf{ASSIGNMENT 2}}}
\author{\textbf{TADIPATRI UDAY KIRAN REDDY}\\\textbf{EE19BTECH11038}}
\maketitle

\section*{\hfil Problem 1}
\begin{figure}[H]
\centering
\begin{circuitikz}[american]
\draw (0,-0.0) to[short, i=$I(x-\Delta x)$] (1,-0.0);
\draw (1,-0.0) to[L,l=$L{\Delta x}$] (3,-0.0);
\draw (3,-0.0) to[R,l=$R{\Delta x}$] (5,-0.0);
\draw (5,-0.0) to[short, i=$I(x)$] (7,-0.0);
\draw (5,-0.0) to[short, i=$I(x-\Delta x)-I(x)$] (5,-1.0);
\draw (4,-1.0) to[R,l=$G{\Delta x}$] (4,-3.0);
\draw (6,-1.0) to[C,l=$C{\Delta x}$] (6,-3.0);
\draw (4,-1.0) to[short] (6,-1.0);
\draw (4,-3.0) to[short] (6,-3.0);
\draw (5,-3.0) to[short] (5,-4.0);
\draw (5,-4.0) to[short] (7,-4.0);
\draw (5,-4.0) to[short] (0,-4.0);

\draw (0, 0) to [open, v=$V(x- {\Delta x})$] (0, -4);
\draw (9, 0) to [open, v=$V(x)$] (9, -4);
\end{circuitikz}
\end{figure}

By KVL.
\begin{gather}
V(s, x - {\Delta}x) = \left(sL+R\right){\Delta}xI(s, x - {\Delta}x) + V(s, x)\\
\implies \frac{V(s, x - {\Delta}x) - V(s, x)}{\Delta x} = \left(sL+R\right)I(s, x - {\Delta}x)
\end{gather}
As $\Delta x -> 0$,
\begin{equation}
	\frac{\partial V(x)}{\partial x} = \left(j{\omega}L+R\right)I(x)
\end{equation}
By KCL, 
\begin{gather}
V(s, x) = \frac{I(s, x - {\Delta}x)-I(s, x)}{(G + sC){\Delta}x}
\end{gather}
As $\Delta x -> 0$,
\begin{equation}
	\frac{\partial I(x)}{\partial x} = \left(j{\omega}C+G\right)V(x)
\end{equation}
From equation (3) and (5), we deduce that,
\begin{gather}
	\frac{\partial^2 V(x)}{\partial x^2} = \left(j{\omega}L+R\right)\left(j{\omega}C+G\right)V(x)\\
\frac{\partial^2 I(x)}{\partial x^2} = \left(j{\omega}L+R\right)\left(j{\omega}C+G\right)I(x)	
\end{gather}
Comparing this with a travelling wave equation we get that,
\begin{equation}
	\gamma = \sqrt{\left(j{\omega}L+R\right)\left(j{\omega}C+G\right)}
\end{equation}
Solution are,
\begin{gather}
	V(x) = V^+e^{-{\gamma}x} + V^-e^{{\gamma}x}\\
	I(x) = \frac{V^-e^{{\gamma}x} - V^+e^{-{\gamma}x}}{Z_0}
\end{gather}
Where $Z_0 = \frac{\left(j{\omega}L+R\right)}{\left(j{\omega}C+G\right)}$
\section*{\hfil Problem 2}
\begin{gather}
	z(t) = x(t)cos({\omega}t) + y(t)sin({\omega}t)\\
	z(t) = \sqrt{x^2(t) + y^2(t)}\left(\frac{x(t)}{\sqrt{x^2(t) + y^2(t)}}cos({\omega}t) + \frac{y(t)}{\sqrt{x^2(t) + y^2(t)}}sin({\omega}t)\right)\\
	z(t) = \sqrt{x^2(t) + y^2(t)}cos({\omega}t - tan^{-1}\left(\frac{y(t)}{x(t)}\right))\\
	\implies A(t) = \sqrt{x^2(t) + y^2(t)}\\
	\implies \phi(t) = - tan^{-1}\left(\frac{y(t)}{x(t)}\right)
\end{gather}
\textbf{We cannot comment on the bandwidth of A(t) and} $\mathbf{\phi}$\textbf{(t) compared to x(t) and y(t).}\\
Say x(t) is m(t)sin($\omega _0$t) and y(t) is  m(t)cos($\omega _0$t) then we can tell that A(t) = m(t) thus band-limited less than both x(t) and y(t) and $\phi$(t) is linear which means band-unlimited take the case where x(t) = y(t) here Phase is constant and thus band-limited.

\end{document}